\documentclass{article}

\title{INFO1111: Computing 1A Professionalism}
\date{Semester 1 2021}
\author{Project 2B}


\usepackage{natbib}
\usepackage{graphicx}
\begin{document}
\maketitle

\begin{center}
    \textbf{Group Number:}
\end{center}

\begin{center}
\begin{tabular}{|l|c|r|}
\hline
  & Full Name & Student ID \\ \hline
1 &           &            \\ \hline
2 &           &            \\ \hline
3 &           &            \\ \hline
4 &           &            \\ \hline    
=======
2 & Mian Wei  & 490038633  \\ \hline
3 &           &            \\ \hline
4 &           &            \\ \hline
\end{tabular}
\end{center}
\thispagestyle{empty}
\newpage

\setcounter{page}{1}
\section{Introduction}
You should remove this text and replace it with your report contents
Overview of the contents, purpose of the report and outcome (i.e. what two non-computing electives were selected for each computing major).


\section{Major Allocation}
write text in here
\begin{center}
\begin{tabular}{|l|c|r|}
\hline
  & Name      & Major                \\ \hline
1 & Student A & Software Development \\ \hline
<<<<<<< Updated upstream
2 & Student C & Information Systems  \\ \hline
3 & Student D & Data Science         \\ \hline
4 & Student D & Computer Science     \\ \hline
=======
2 & Mian Wei  & Data Science         \\ \hline
3 &           & Information Systems  \\ \hline
4 & Student D & Data Science         \\ \hline
\end{tabular}
\end{center}

\section{Recommendations}

\subsection{Computer Science}


\subsection{Data Science}
% MATH1005 Statistical Thinking with Data
% QBUS2820: Predictive Analytics

In data-rich world, students majored in data science are facing the huge amount of data, and sometimes they do not know how to analyze data and use data to make prediction (especially in business field, to help make comparisons to competition, analyze the market, and ultimately, make recommendations of when and where your product or services will sell best). They should have statistical thinking and data analysis ability, so that they will have better development. In my opinion, MATH1005 in Statistics area and QBUS2820 in Business Analytics area are good choices as electives.

The full name of MATH1005 is Statistical Thinking with Data, this unit concentrates on basic statistical concepts, including experimental design, exploratory data analysis, sampling, and sampling and tests of significance. About QBUS2820—Predictive Analytics, this unit introduces different techniques of data analysis and modelling that can be applied to traditional and non-traditional problems in a wide range of areas including stock forecasting, fund analysis, asset allocation, equity and fixed income option pricing, consumer products, as well as consumer behavior modelling (credit, fraud, marketing). The forecasting techniques covered in this unit are useful for preparing individual business forecasts and long-range plans. The unit takes a practical approach with many up-to-date datasets used for demonstration in class and in the assignments.

Statistics is used in every aspect of life, such as in data science, robotics, business, sports, weather forecasting, and much more. ... Additionally, statistics help in learning mathematical concepts better. This is how statistics can be used in each aspect of real life. Statistical thinking is the ability to align one’s thoughts with the fundamental ideas of statistics, allowing the person to make better decisions in any given situation. Why do people need statistical thinking? Actually, human intuition often tries to answer the same questions that we can answer using statistical thinking, but often gets the answer wrong. For example, in recent years most Americans have reported that they think that violent crime was worse compared to the previous year (Pew Research Center, 2020). However, a statistical analysis of the actual crime data shows that in fact violent crime has steadily decreased since the 1990’s. Statistical thinking provides us with the tools to more accurately understand the world and overcome the fallibility of human intuition. The core of MATH1005 is Statistical thinking, this course will teach students how to process data with R and critique the use of statistics in media and research papers, with attention to confounding and bias, which are powerful skills in data science students' professional careers (like statistician, data analyst and data scientist). Statisticians think in terms of probabilities and variation and search for the data that might support specific contentions and assess their validity. For example, in assessing the safety of different forms of travel, statisticians might estimate the risk probabilities associated with the various alternatives and tailor these to their specific situations, such as driving skill, available air travel options, and weather conditions (“The excitement of a career in statistics” 2012).

In business, predictive analytics is becoming more and more important for company development. To be more specific, predictive analytics has captured the support of wide range of organizations, with a global market projected to reach approximately \$10.95 billion by 2022, growing at a compound annual growth rate (CAGR) of around 21 percent between 2016 and 2022, according to a 2017 report issued by Zion Market Research. Predictive models will exploit patterns found in historical and transnational data to identify risks and opportunities. Models capture relationships among many factors to allow assessment of risk or potential associated with a particular set of conditions, guiding decision-making for candidate transactions (Coker, Frank, 2014). For example, in 2004, Walmart mined transaction data in its stores to understand buying habits at certain points in time. They found that right before hurricanes hit, strawberry Pop-Tart sales rose by seven times along with beer. There are six main applications of predictive analytics in business: customer targeting, churn prevention, sales forecasting, quality improvement, risk assessment, financial modeling. These six applications help students get better understanding with what they will do when they select related jobs (like project manager, IT systems analyst and operations analyst). To develop ability of predictive analysis, QBUS2820, in this unit, students will be systematically trained how to select and use the appropriate technique to analyse the structure of multivariate data, apply multivariate data techniques using a training data set to predict classifications for real data, understand the characteristics of time-series data in order to analyse real business data of this form and select and use an appropriate technique to predict the future behavior of business variables of interest, including the prediction of discrete outcomes.

In data science, statistical thinking and predictive analytics are essential and valuable. What careers can students pursue with statistical thinking and predictive analytics? In general, there is great demand in business area. If students enjoy evaluating and analysing data, creating solutions, a career as a business analyst could be for them. Business analyst will work within an organisation, helping to manage, change and plan for the future in line with their goals. This could be for one specific project, or as a permanent feature of the organisation. Business analysts need to understand the current organisational situation, identify future needs and create solutions to help meet those needs. Referring to 2021 average salary for a Business Analyst in Australia, the base salary is between AU \$59k and AU \$112k per year, and the average salary for a Business Analyst is AU \$82,241 per year. If students are higly analytical and have strong mathematical skills, data analyst tends to be better choice for them. Data analysts are in high demand across all sectors, such as finance, consulting, manufacturing, pharmaceuticals, government and education, and the 2021 average salary in Australia for a Data Analyst is AU \$71,994 per year.

In conclusion, MATH1005 and QBUS2820 will development data science students' statistical thinking and predictive analytics, which are not only beneficial for learning data science but useful in careers. Students can rely on these two points to be more competitive in the future careers and maximize the benefits of the company and the society.

\subsection{Information Systems}


\subsection{Software Development}

\section{Conclusion}
``I always thought something was fundamentally wrong with the universe'' \citep{adams1995hitchhiker}

\bibliographystyle{plain}
\bibliography{references}
=======
``I always thought something was fundamentally wrong with the universe\cite{Burridge_2018a}''
\bibliography{references}{}
\bibliographystyle{apalike}
% \bibliographystyle{ieeetran}

% https://www.sydney.edu.au/handbooks/science/subject_areas_nz/statistics_descriptions.shtml

% https://www.sydney.edu.au/handbooks/interdisciplinary_studies/subject_areas/subjects_ac/business_analytics.shtml

% Poldrack, C. (n.d.). Statistical Thinking for the 21st Century. Retrieved from https://web.stanford.edu/group/poldracklab/statsthinking21/introduction.html#what-is-statistical-thinking

% Gramlich, J. (2020, November 23). What the data says (and doesn't say) about crime in the United States. Retrieved from https://www.pewresearch.org/fact-tank/2020/11/20/facts-about-crime-in-the-u-s/

% The excitement of a career in statistics. (n.d.). Retrieved from https://stattrak.amstat.org/2012/01/31/excitingstatisticscareer/

% What is predictive analytics? (2021, May 13). Retrieved from https://www.predictiveanalyticstoday.com/what-is-predictive-analytics/#:~:text=Predictive%20analytics%20is%20the%20branch,to%20make%20predictions%20about%20future.

% 6 business applications of predictive analytics. (n.d.). Retrieved from https://www.neuraldesigner.com/blog/6_Applications_of_predictive_analytics_in_business_intelligence#ChurnPrevention

% Edwards, J. (2019, August 16). Predictive analytics: Transforming data into future insights. Retrieved from https://www.cio.com/article/3273114/what-is-predictive-analytics-transforming-data-into-future-insights.html
\end{document}
