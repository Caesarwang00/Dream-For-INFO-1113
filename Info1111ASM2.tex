\documentclass{article}

\title{INFO1111: Computing 1A Professionalism}
\date{Semester 1 2021}
\author{Project 2B}


\usepackage{natbib}
\usepackage{graphicx}
\begin{document}
\maketitle

\begin{center}
    \textbf{Group Number:}
\end{center}

\begin{center}
\begin{tabular}{|l|c|r|}
\hline
  & Full Name & Student ID \\ \hline
1 &           &            \\ \hline
2 &           &            \\ \hline
3 &           &            \\ \hline
\end{tabular}
\end{center}
\thispagestyle{empty}
\newpage

\setcounter{page}{1}
\section{Introduction}
You should remove this text and replace it with your report contents
Overview of the contents, purpose of the report and outcome (i.e. what two non-computing electives were selected for each computing major).


\section{Major Allocation}
write text in here
\begin{center}
\begin{tabular}{|l|c|r|}
\hline
  & Name      & Major                \\ \hline
1 & Student A & Software Development \\ \hline
2 & Student C & Information Systems  \\ \hline
3 & Student D & Data Science         \\ \hline
\end{tabular}
\end{center}

\section{Recommendations}

\subsection{Computer Science}

I chose Medicine and Agriculture.The reasons are simple, but mainly include two things. First, the two courses cover a wide range of subjects, and second, they are both practical subjects.
These two elective courses have a lot of applications, whether it is hospital medical system, or automated farm management can not be separated from computer science.The hospital health care system is faced with a large number of patients every day, including the elderly, young and children.The level of education of these patients is different, which makes the procedure readability a requirement.So the person writing the program naturally has to have a certain amount of medical knowledge, because they have to have that knowledge before they can explain it in plain English to others.And then there's Agriculture. For example, when you're writing programs for automated cotton growing, when to sprinkle water, what humidity and temperature to keep in the greenhouse, and so on. These are things that need to be taken into account and written into the program, which brings up the need to have an understanding of agriculture.
When it comes to medicine, the first thing we need to know is chemistry. Medicine themselves are the product of the application of chemistry in life sciences.So I choose Chemistry (CHEM1011) as my elective course. Although chemistry plays an important role in medicine, how can it help students of computer science? In modern hospitals, more and more automated procedures have been added. First, the appointment system, then the medical examination, and finally the purchase of drugs, all these procedures can be automated. In the process of automation, programmers who write programs inevitably need to be familiar with and understand certain medical knowledge.And the beginning of all this, in my opinion, is chemistry. Since chemistry is so important, are there any special requirements for studying chemistry?According to the University of Sydney website,There is no assumed knowledge of chemistry for this unit of study but students who have not completed HSC Chemistry (or equivalent) are strongly advised to take the Chemistry Bridging Course (offered in February).(USTD N.D.)
Next, I'd like to talk about the electives in agriculture.First of all, we need to define a target, the target is what kind of crops to breed, and the best way to choose is to consider their economic value, and if we need to analyze whether a crop has commercial value.I think ECON1002 is a really good choice.The second step is to understand how computer science as a major applies to the field.First of all, different crops have different growth cycles, and different periods of time require different amounts of water, temperature and sunlight.These factors are variables, and the control of variables happens to be an important and frequently used part of computer science.Using the knowledge in computer science to establish a set of crop management system is a quite feasible scheme.Since the first step is to understand which crops have high commercial value, it is necessary to learn ECON1002.First of all, the requirements for the course,Students enrolled in this unit have an assumed knowledge equal to or exceeding 70 or higher in HSC Mathematics (or equivalent), or 35 or higher in HSC Mathematics Extension 1 (or equivalent), or 35 or higher in HSC Mathematics Extension 2 (or equivalent).(USTD N.D.)
Finally, I will summarize the above.My chosen fields are agriculture and medicine.In these two aspects, the general direction of the application of computer science is the same, that is, the automation part of modernization. Automation, as the name implies, is to replace manual labor with machines, and machines are run in accordance with the program, and the program is the way to apply computer science.However, there are some differences in the professional courses of the two fields. In medical treatment, I hope to learn some knowledge about drugs first, because drugs are the second most important thing in hospitals besides doctors.However, in terms of agriculture, I want to start from the economic aspect, because I need to know what agricultural products I want to know, that is, which agricultural products with high economic benefits are. In the end, I chose ECON1002 and CHEM1011.
To make my argument more convincing, I'm going to give you some examples.The first is to prove the importance of chemistry in medicine, the essence of life is the complex chemical reaction, and the life of death is the end of the reaction, the effect of drugs and medical is to add the necessary by this chemical elements or add catalyst, add can save lives and correct the wrong drug will direct the end of the reaction, also is death.This makes it impossible for a programmer to be sloppy, a single mistake, a pill of the wrong kind, could be a disaster.Therefore, chemistry is a must to learn, whether it is in the supplement of professional knowledge, or respect for the lives of others, this is a must.Followed by the choice of crops, through some of the information on the Internet search, I found that in addition to planting conditions, planting quantity, the yield under the same area, the worldwide popularity, the margins when selling, the cost of land, the cost of workers machinery must be taken into account, so I will face the choice of diversification in choosing targets.I still can't make an accurate judgment or make a decision, which makes me have to study and understand the economic market, then analyze the data and make my choice.


\subsection{Data Science}


\subsection{Information Systems}


\subsection{Software Development}

\section{Conclusion}
``I always thought something was fundamentally wrong with the universe'' \citep{adams1995hitchhiker}

\bibliographystyle{plain}
\bibliography{references}

https://www.scott-clark.com/blog/6-common-uses-for-computers-in-healthcare/
https://www.agriculture.gov.au/ag-farm-food/crops/cotton
https://www.sydney.edu.au/handbooks/science/subject_areas_ae/chemistry_descriptions.shtml
https://www.sydney.edu.au/handbooks/arts/subject_areas_eh/economics_table.shtml
https://www.theatlantic.com/magazine/archive/1909/01/modern-chemistry-and-medicine/529875/
https://www.visualcapitalist.com/the-worlds-most-valuable-cash-crop/
\end{document}
