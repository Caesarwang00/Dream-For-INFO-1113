\documentclass[12pt]{article}

\title{INFO1111: Computing 1A Professionalism}
\date{Semester 1 2021}
\author{Project 2B}

\usepackage{graphicx}
\begin{document}
\maketitle

\begin{center}
    \textbf{Group Number:}
\end{center}

\begin{center}
\begin{tabular}{|l|c|r|}
\hline
  & Full Name & Student ID \\ \hline
1 &           &            \\ \hline
2 &           &            \\ \hline
3 &           &            \\ \hline
4 &           &            \\ \hline
\end{tabular}
\end{center}
\thispagestyle{empty}
\newpage

\setcounter{page}{1}
\section{Introduction}
You should remove this text and replace it with your report contents
Overview of the contents, purpose of the report and outcome (i.e. what two non-computing electives were selected for each computing major).


\section{Major Allocation}
write text in here
\begin{center}
\begin{tabular}{|l|c|r|}
\hline
  & Name      & Major                \\ \hline
1 & Student A & Software Development \\ \hline
2 & Student B & Data Science        \\ \hline
3 & Student C & Information Systems  \\ \hline
4 & Student D & Data Science         \\ \hline
\end{tabular}
\end{center}

\section{Recommendations}

\subsection{Computer Science}


\subsection{Data Science}


\subsection{Information Systems}
% QBUS3350 Project Planning and Management
% BUSS1040 Economics for Business Decision Making

omputer system students will focus more on data analysis and processing, and use programming languages for big data processing. For students of Information systems, if the analytical skills and business skills are combined, it may bring more advantages in the future competition. Based on the above thoughts, I think QBUS3350 and BUSS1040 in the Business and Commerce area are suitable as electives.

The full name of QBUS3350 is Project Planning and Management, this course focuses on teaching students the ability to plan, implement and manage activities to achieve specific organizational goals, and discuss the challenges that project managers may encounter. BUSS1040 is a compulsory course for the Department of Finance, which full name is called Economics for Business Decision Making, it teaches students basic business knowledge and decision-making, as well as interaction in the market. Students majoring in information systems have certain data analysis skills, advanced mathematical and logical thinking skills and the ability to write codes, and they will have the ability to operate, manage and make decisions on the system in the future. I will analyze the advantages of the two elective subjects to explain why it is suitable for students of information systems to study.

QBUS3350 course will teach students the ability to plan, implement and manage activities, under normal circumstances, students majoring in information systems will be more inclined to perform database operation and maintenance management. After learning this course, they can participate in the product planning process. For any database product, performance is particularly important. Databases are usually designed by architects, and unreasonable basic database structure design can lead to hidden dangers in system performance. After the amount of data increases, the performance and stability of the database begin to decline, which will directly affect the response speed and user experience of the product. If personnel with organizational skills and operation and maintenance capabilities are planned in the early stage of the design, it will save a lot of time and money for maintaining the database in the later stages.




\subsection{Software Development}

\section{Conclusion}
``I always thought something was fundamentally wrong with the universe\cite{Burridge_2018a}''
\bibliography{references}{}
\bibliographystyle{apalike}
% \bibliographystyle{ieeetran}


% https://www.sydney.edu.au/handbooks/interdisciplinary_studies/units_of_study/tables_dh/finance_descriptions.shtml
% https://www.navisite.com/blog/when-database-refactoring-is-a-smart-decision/

\end{document}

